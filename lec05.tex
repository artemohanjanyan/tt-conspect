\section{\texorpdfstring{Логика второго порядка и полиморфизм}{Second-order logic and polymorphism}}

Обычное $\lambda$-исчисление позволяет слишком много,
просто-типизированное "--- слишком мало.
Например, в просто типизированном $\lambda$-исчислении нельзя задать общий тип для чисел Чёрча.
Хотелось бы найти золотую середину.

С этого раздела начинается поиск оптимальной типовой системы,
которая удовлетворяла бы все наши потребности,
и могла бы быть основой для языка программирования.

\subsection{\texorpdfstring{Интуиционистское исчисление предикатов второго порядка}{Second order intuitionistic logic}}

\begin{definition}[грамматика ИИП второго порядка]
    \begin{bnf}
    \[
        \Phi ::= (\Phi) | p | \Phi \to \Phi | \forall p . \Phi \color{gray}
            \underbrace{| \exists p . \Phi | \bot | \Phi \with \Phi | \Phi \vee \Phi}_{\text{не существенные}}
    \]
    \end{bnf}
\end{definition}

\begin{definition}[правила вывода в ИИП второго порядка]
    К правилам обычного ИИВ добавляются правила вывода для квантора всеобщности:
    \[
        \infer[(p \notin \FV(\Gamma))]{\Gamma \vdash \forall p . \varphi}{\Gamma \vdash \varphi} \qquad
        \infer{\Gamma \vdash \varphi \left[p \coloneqq \sigma\right]}{\Gamma \vdash \forall p . \varphi}
    \]
    И для квантора существования:
    \[
        \infer{\Gamma \vdash \exists p . \varphi}{\Gamma \vdash \varphi \left[p \coloneqq \psi\right]} \qquad
        \infer[(p \notin \FV(\Gamma, \psi))]{\Gamma \vdash \psi}{\Gamma \vdash \exists p . \varphi && \Gamma, \varphi \vdash \psi}
    \]
    где $\FV(\Gamma)$ "--- множество свободных переменных всех выражений из $\Gamma$.
\end{definition}

Напомним, что в логике второго порядка переменные под кванторами соответствуют любым выражениям,
а не только значениям из предметного множества.

Последние четыре связки можно выразить через первые.
$\bot$ "--- это критерий противоречивости,
из него можно вывести любое утверждение. Это можно выразить так:
\[
    \bot \equiv \forall p . p
\]
Такое выражение естественным образом соответствует нашим требованиям,
так как пользуясь правилом вывода для квантора всеобщности из такого $\bot$ можно вывести любое выражение.

Приведём выражение для конъюнкции:
\[
    \varphi \with \psi \equiv \forall a . ((\varphi \to \psi \to a) \to a)
\]
Его можно читать как "<если всё, что выводимо из $\set{\varphi,\psi}$, выполняется,
то тогда выполняется $\varphi\with\psi$">.
Также для такого выражения можно проверить, что правила вывода конъюнкции действительно выводятся через такое определение.
Например, выведем $\varphi$ из $\varphi\with\psi$:
\[
    \infer{\Gamma \vdash \varphi}
    {
        \infer{\Gamma \vdash (\varphi\to\psi\to\varphi)\to\varphi}
            {\Gamma \vdash \forall \gamma.(\varphi\to\psi\to\gamma)\to\gamma}
        &&
        \infer{\Gamma\vdash\varphi\to\psi\to\varphi}
        {\infer{\Gamma,\varphi\vdash\psi\to\varphi}
        {\infer{\Gamma,\varphi,\psi\vdash\varphi}
        {}}}
    }
\]

Выражение для дизъюнкции:
\[
    \varphi \vee \psi \equiv \forall a . (\varphi \to a) \to (\psi \to a) \to a
\]
Формулу для дизъюнкции можно читать как "<$\varphi \vee \psi$ выполняется,
если любое $a$, которое выводимо как из $\varphi$, так и из $\psi$, выполняется">.
Ещё её можно понимать как следствие из формулы де Моргана $\varphi\vee\psi=\neg(\neg\varphi\with\neg\psi)$.

Формула для квантора существования получается из того, что $\exists x . \tau = \neg \forall x . \neg \tau$:
\[
    \exists x . \tau \equiv \forall a . (\forall x . \tau \to a) \to a
\]

\subsection{\texorpdfstring{Система $F$}{System F}}
\begin{definition}[тип в системе $F$]
\[
    \tau =
    \begin{cases}
        \alpha, \beta, \gamma, \ldots & \text{(атомарный тип)} \\
        \tau \to \tau \\
        \forall \alpha . \tau & \text{($\alpha$ "--- переменнная)}
    \end{cases}
\]
\end{definition}

\begin{definition}[система $F$]
Грамматика выражения в системе $F$:
    \begin{bnf}
    \[
        \mathbf\Lambda ::= x | \lambda p^\alpha . \mathbf\Lambda | \mathbf\Lambda \mathbf\Lambda | (\mathbf\Lambda)
        | \Lambda \alpha . \mathbf\Lambda | \mathbf\Lambda \tau
    \]
    \end{bnf}%
    $\Lambda \alpha . \mathbf\Lambda$ "--- полиморфная абстракция, $\mathbf\Lambda \tau$ "--- применение типа.
    Правила вывода:
    \inferspacing
    \begin{gather*}
        \infer[(x \notin \mathrm{dom}(\Gamma))]{\Gamma, x : \sigma \vdash x : \sigma}{} \\
        \infer{\Gamma \vdash MN : \sigma}{\Gamma \vdash M : \tau \to \sigma && \Gamma \vdash N : \tau} \qquad
        \infer[(x \notin \mathrm{dom}(\Gamma))]
                {\Gamma \vdash \lambda x^\tau . M : \tau \to \sigma}{\Gamma, x : \tau \vdash M : \sigma} \\
        \infer[(\alpha \notin \FV(\Gamma))]{\Gamma \vdash \Lambda \alpha . M : \forall \alpha . \sigma}{\Gamma \vdash M : \sigma} \qquad
        \infer{\Gamma \vdash M \tau : \sigma [\alpha := \tau]}{\Gamma \vdash M : \forall \alpha . \sigma}
    \end{gather*}
\end{definition}

Полиморфная абстракция "--- это явное указание того, что вместо каких-то типов мы можем подставить любые выражения.

\begin{example} Левая проекция:
    \begin{center}
    \begin{tabular}{l l l} \toprule
        & Просто типизированное $\lambda$-исчисление & Система $F$ \\ \midrule
        Тип & $\pi_1:\alpha\with\beta\to\alpha$ & $\pi_1:\forall \alpha . \forall \beta . \alpha \with \beta \to \alpha$ \\
        Выражение & $\pi_1 = \lambda p . p \comb T$ & $\pi_1 = \Lambda \alpha . \Lambda \beta . \lambda p^{\alpha\with\beta} .
                p \alpha \comb T$
        \\ \bottomrule
    \end{tabular}
    \end{center}
В системе $F$ явно указывается, что типы элементов пары могут быть любыми, как и в типе проекции, так и в её выражении.
\end{example}

\begin{definition}[$\beta$-редукция в $F$] \ 
    \begin{enumerate}
        \item Типовая редукция: $\left(\Lambda \alpha . M^\sigma\right) \tau \to_\beta M[\alpha:=\tau] : \sigma[\alpha := \tau]$
        \item Классическая $\beta$-редукция: $\left(\lambda x^\sigma.M\right)^{\sigma\to\tau} X \to_\beta M [x:=X] : \tau$
    \end{enumerate}
\end{definition}

Выразим несколько функций:
\begin{enumerate}
    \item Левая инъекция:
    \[
        \inj_1 = \Lambda \varphi . \Lambda \psi . \lambda x^\varphi .
            \underline{\Lambda \alpha . \lambda f^{\varphi\to\alpha}.\lambda g^{\psi\to\alpha}.f x}
    \]
    Можно заметить, что подчёркнутый терм имеет тип $\varphi \vee \psi$.
    \item Для ясности приведём (довольно бесполезное) выражение $\mathtt{case}$:
    \[
        \case{T^{\varphi\vee\psi}}{A^{\varphi\to\alpha}}{B^{\psi\to\alpha}} = T \alpha A B
    \]
    Результат всего выражения имеет тип $\alpha$, который мы явно передали в выражение абстрактного типа данных $T$.
    \item Конструктор пары:
    \[
        \pair{a^\varphi, b^\psi} = \Lambda \varphi . \Lambda \psi . \lambda a^\varphi . \lambda b^\psi .
            \Lambda \alpha . \lambda f^{\varphi\to\psi\to\alpha} . f a b
    \]
    \item Правая проекция:
    \[
        \pi_2 = \Lambda \varphi . \Lambda \psi . \lambda p^{\varphi \with \psi}.p\psi\left(\lambda x^\varphi\lambda y^\psi.y\right)
    \]
\end{enumerate}

В системе $F$ можно задать общий тип для чёрчевского нумерала: $\forall \alpha.(\alpha\to\alpha)\to\alpha\to\alpha$.
Тип $\alpha$ можно менять в зависимости от того, как нумерал используется.

\begin{theorem}[изоморфизм Карри-Ховарда для системы $F$]
    $\Gamma \vdash_F M :\tau$ т.и.т.т., когда $\abs{\Gamma} \vdash \tau$ в интуиционистском исчислении предикатов второго порядка.
\end{theorem}

Доказательство изоморфизма для системы $F$ абсолютно аналогичное.
Рассмотрим, например, следующее верное утверждение в ИИП второго порядка: "<из лжи следует что угодно">, или
$\forall \beta . \bot \to \beta \equiv \forall \beta . (\forall \alpha.\alpha) \to \beta$.
Если мы выпишем $\lambda$-выражение с таким типом, то станет очевидно, как доказывать утверждение. Пример такого выражения:
\[
    \Lambda \beta . \lambda a^{\forall \alpha.\alpha} . a \beta :
    \forall \beta . (\forall \alpha.\alpha) \to \beta
\]
Вывод типа выражения, как и раньше, легко выписывается исходя из самого выражения,
так как $\lambda$-выражение каждого из видов, описанных в грамматике, можно получить только одним правилом.
\[
    \infer{\vdash \Lambda \beta . \lambda a^{\forall \alpha.\alpha} . a \beta : \forall \beta . (\forall \alpha.\alpha) \to \beta}
    {   \infer{\vdash \lambda a^{\forall \alpha.\alpha} . a \beta : (\forall \alpha.\alpha) \to \beta}
        {   \infer{a : \forall \alpha . \alpha \vdash a \beta : \beta}
            {   \infer{a : \forall \alpha.\alpha \vdash a : \forall \alpha.\alpha}{}
            }
        }
    }
\]
Другое $\lambda$-выражение с таким же типом это
$\Lambda \beta . \lambda a^{\forall \alpha . \alpha} . a ((\forall \alpha.\alpha) \to \beta) a$.
%\[
%    \infer{\vdash \Lambda \beta . \lambda a^{\forall \alpha . \alpha} . a ((\forall \alpha.\alpha) \to \beta) a :
%            \forall \beta . (\forall \alpha.\alpha) \to \beta}
%    {   \infer{\vdash \lambda a^{\forall \alpha . \alpha} . a ((\forall \alpha.\alpha) \to \beta) a :
%                (\forall \alpha.\alpha) \to \beta}
%        {   \infer{a : \forall \alpha . \alpha \vdash a ((\forall \alpha.\alpha) \to \beta) a : \beta}
%            {   \infer{a : \forall \alpha . \alpha \vdash a ((\forall \alpha.\alpha) \to \beta)
%                        : (\forall \alpha . \alpha) \to \beta}
%                {   \infer{a : \forall \alpha.\alpha \vdash a : \forall \alpha.\alpha}{}
%                }
%            &&  \infer{a : \forall \alpha . \alpha \vdash a : \forall \alpha.\alpha}{}
%            }
%        }
%    }
%\]

\begin{theorem}
    $F$ \hyperref[strong-normalization]{сильно нормализуема}.
\end{theorem}
