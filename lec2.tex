\section{Вторая лекция}

\begin{definition}[Ромбовидное свойство (diamond)]
    $G$ обладает ромбовидным свойством, если какие бы ни были $a$, $b$, $c$, что $aGb$, $aGc$, $b \ne c$, найдётся $d$: $bGd$ и $cGd$.
\end{definition}

\begin{example}
    $(<)$ на натуральных числах обладает ромбовидным свойством.
    $(>)$ на натуральных числах не обладает ромбовидным свойством.
\end{example}

$\beta$-редукция не обладает ромбовидным свойством.
\begin{example}
    \begin{gather*}
        a = (\lambda x . x x)(Ia) \\
        a \rightarrow_{\beta} (Ia)(Ia) = b\\
        a \rightarrow_{\beta} (\lambda x . x x) a = c \\
        b \rightarrow_{\beta} (Ia)a \rightarrow_{\beta} aa \\
        b \rightarrow_{\beta} a(Ia) \rightarrow_{\beta} aa \\
        c \rightarrow_{\beta} aa
    \end{gather*}
    Нет $d$, что $b \rightarrow_{\beta} d$ и $c \rightarrow_{\beta} d$.
\end{example}

\begin{theorem}[Чёрча-Россера]
    $\beta$-редуцируемость обладает ромбовидным свойством.
\end{theorem}

\begin{lemma}
    Если $R$ обладает ромбовидным свойством, то $R^{*}$ обладает ромбовидным свойством.
\end{lemma}

\begin{proof} Упражнение % TODO
    \begin{enumerate}
        \item $M_{1}RN_{1}$ и $M_{1}RM_{2}...M_{n-1}RM_{n}$ $\Rightarrow$ есть $N_{2}...N_{n}$: \\
            $N_{1}RN_{2}...N_{n-1}RN_{n}$ и $M_{n}RN_{n}$.
        \item Покажем ромбовидное свойство.
    \end{enumerate}
\end{proof}

\begin{definition}[Параллельная $\beta$-редукция]
    $A \rightrightarrows_{\beta} B$
    \begin{enumerate}
        \item $A =_{\beta} B$, то $A \rightrightarrows_{\beta}B$
        \item $A \rightrightarrows_{\beta} B$, то $\lambda x.A \rightrightarrows_{\beta} \lambda x . B$
        \item $P \rightrightarrows_{\beta} Q$ и $R \rightrightarrows_{\beta} S$, то $PR \rightrightarrows_{\beta} QS$
        \item $(\lambda x . P) Q \rightrightarrows_{\beta} R_{[x:=S]}$, если 
            $P \rightrightarrows_{\beta}R$ и $Q \rightrightarrows_{\beta} S$.
    \end{enumerate}
\end{definition}

\begin{statement}
    $(\rightrightarrows_{\beta})$ обладает ромбовидным свойством.
\end{statement}

\begin{proof}
    Упражнение % TODO
\end{proof}

\begin{statement}
    Если $A \rightarrow_{\beta} B$, то $A \rightrightarrows_{\beta} B$.
\end{statement}

\begin{statement}
    Если $A \rightrightarrows_{\beta} B$, то $A \twoheadrightarrow_{\beta} B$.
\end{statement}

\begin{proof}
    Упражнение % TODO
\end{proof}

При этом, обратное не всегда верно.

\begin{example}
    \begin{gather*}
        (\lambda x . x x) (\lambda x . x x x) \twoheadrightarrow_{\beta} (\lambda x . x x x)(\lambda x . x x x)(\lambda x . x x x) \\
        (\lambda x . x x) (\lambda x . x x x) \cancel{\rightrightarrows_{\beta}} (\lambda x . x x x)(\lambda x . x x x)(\lambda x . x x x)
    \end{gather*}
\end{example}

\begin{statement}
    Из предыдущих двух утверждений следует $(\rightarrow_{\beta})^{*} = (\rightrightarrows_{\beta})^{*}$.
\end{statement}

Теорема Чёрча-Россера следует из приведённых утверждений.

\begin{statement}[Следствие] %следствие TODO
    Нормальная форма для $\lambda$-выражения единственна, если существует.
\end{statement}

\begin{theorem}[Тезис Чёрча]
    Если функция вычислима с помощью механического аппарата, то она вычислима с помощью $\lambda$-выражения.
\end{theorem}

\paragraph{Порядок редукции}
\begin{definition}
    \begin{align*}
        K &= \lambda x \lambda y . x \\
        I &= \lambda x . x \\
        S &= \lambda x y z . x z (y z)
    \end{align*}
\end{definition}
$I$ выражается через $S$ и $K$: $I = S K K$.

\begin{statement}
    Пусть $A$ "--- замкнутое $\lambda$-выражение. Тогда найдётся выражение $T$, состоящее только из $S$, $K$, что $A =_{\beta}T$.
\end{statement}

\begin{example}
    тут какой-то пример с омегой % TODO
\end{example}

\begin{definition}[Нормальный порядок редукции]
    Нормальным порядком редукции называется редукция самого левого $\beta$-редекса.
\end{definition}
"<Ленивые вычисления"> (ну, почти, в них есть меморизация)

\begin{definition}[Аппликативный порядок редукции]
    Самый левый из самых вложенных.
\end{definition}
"<Энергичные вычисления">

\begin{statement}
    Если нормальная форма существует, она может быть достигнута нормальным порядком редукции.
\end{statement}

\paragraph{Изоморфизм Карри-Ховарда} \mbox{} \\

Хотим построить логику.
Скажем, что есть импликация, обозначим $(\supset)$. Введём M.P. и правила:
\begin{enumerate}
    \item $A \supset A$
    \item $(A \supset (A \supset B)) \supset (A \supset B)$
    \item $A =_{\beta} B$, тогда $A \supset B$
\end{enumerate}

Заметим:
\[
    Y_{\supset a} \equiv Y (\lambda t . t \supset a) =_{\beta} Y (\lambda t . t \supset a) \supset a
\]

\begin{align*}
    &Y_{\supset a} \supset Y_{\supset a} \\
    &Y_{\supset a} \supset (Y_{\supset a} \supset a) \\
    &(Y_{\supset a} \supset Y_{\supset a} \supset a) \supset (Y_{\supset a} \supset a) \\
    &Y_{\supset a} \supset a \\
    &(Y_{\supset a} \supset a) \supset Y_{\supset a} \\
    &Y_{\supset a} \\
    &a
\end{align*}

Так можно доказать любое $a$. Парадокс. Нехорошо.

Рассмотрим интуиционистское исчисление высказываний.
\begin{align*}
    &\frac{}{\Gamma, \varphi \vdash \varphi} \\
    &\frac{\Gamma, \varphi \vdash \psi}{\Gamma \vdash \varphi, \varphi} \text{ (Введение)} \\
    &\frac{\Gamma \vdash \varphi \rightarrow \psi \qquad \Gamma \vdash \varphi}{\Gamma \vdash \psi} \text{ (Удаление)}
\end{align*}

тут было докво

\begin{theorem}
    Импликационный фрагмент интуиционистского исчисления высказываний полон в моделях Крипке
\end{theorem}
