% Филипп Вадлерasdasd
\subsection{\texorpdfstring{Линейные типы}{Linear types}}
%Пусть у нас есть некоторые набор переменных $x$, $y$, $z$ и некоторый набор термов $s$, $t$, $u$, $v$, $w$.

Определим грамматику для нашего $\lambda$-исчисления:
\begin{bnf}
\begin{alignat*}{2}
	s ::= &\ x \\
			  | &\ \lambda\pair{x}.s | s\pair{s} \\
			  | &\ \oc s | \caseb{s}{\oc x \to s} \\
			  | &\ \pair{s,s} | \caseb{s}{\pair{x,x} \to s} \\
			  | &\ \ppair{s,s} | \fst \pair{s} | \snd \pair{s} \\
			  | &\ \inl \pair{s} | \inr \pair{s}|
                    \casec{s}{\inl \pair{x} \to s}{\inr \pair{x} \to s}
\end{alignat*}
\end{bnf}
Будем типизировать это $\lambda$-исчисление линейными высказываниями. Выпишем аксиомы:
\begin{@empty}
\inferspacing
\begin{gather*}
	\infer[]
		{[\Gamma]\vdash \oc s : \oc A}
		{[\Gamma]\vdash s : A} \qquad
	\infer[]
		{\Gamma, \Delta\vdash \caseb{s}{\oc x \to t} : B}
		{\Gamma\vdash s : \oc A && \Delta, [x : A]\vdash t : B} \\
	\infer[]
		{\Gamma, \Delta\vdash \pair{s, t} : A \otimes B}
		{\Gamma \vdash s : A && \Delta\vdash t : B} \qquad
	\infer[]
		{\Gamma, \Delta \vdash \caseb{s}{\pair{x, y} \to t} : C}
		{\Gamma\vdash s : A \otimes B && \Delta, \pair{x : A}, \pair{y : B} \vdash t : C} \\
	\infer[]{\Gamma \vdash \ppair{s, t} : A \with B}{\Gamma \vdash s : A && \Gamma \vdash t : B} \qquad
	\infer[]{\Gamma \vdash \fst \pair{s} : A}{\Gamma \vdash s : A \with B} \qquad
	\infer[]{\Gamma \vdash \snd \pair{s} : B}{\Gamma \vdash s : A \with B} \\
	\infer[]{\Gamma, \lambda\pair{x}.u : A \multimap B}{\Gamma, \pair{x : A} \vdash u : B} \qquad
	\infer[]{\Gamma, \Delta \vdash  s \pair{t} : B}{\Gamma \vdash s : A \multimap B && \Delta \vdash t : A} \\
	\infer[]{\Gamma \vdash \inl \pair{s} : A \oplus B}{\Gamma \vdash s : A} \qquad
	\infer[]{\Gamma \vdash \inr \pair{s} : A \oplus B}{\Gamma \vdash s : B} \\
	\infer[]
		{\Gamma, \Delta \vdash \casec{s}{\inl \pair{x} \to t}{\inr \pair{y} \to v} : C}
		{\Gamma \vdash s : A \oplus B && \Delta, \pair{x : A} \vdash t : C && \Delta, \pair{y : B} \vdash v : C}
\end{gather*}
\end{@empty}
Вложим просто типизируемое $\lambda$-исчисление в линейные типы.
\begin{align*}
	\lambda x.s &= \lambda\pair{x'}.\caseb{x'}{\oc x \to s} \\
	st          &= s\pair{\oc t}
\end{align*}
\begin{example}[Ломаем линейные типы]
	Пусть у нас есть выражения $f$, $g$  типа $A \multimap B$.
	Возьмем $h = \lambda x.\pair{f\pair{x}, g\pair{x}} : \oc A \multimap B \otimes B$.
	Докажем, что $h$ действительно существует в линейных типах:

	\[
		\infer[]{[f : A \multimap B], [g : A\multimap B]\vdash \lambda x.\pair{f\pair{x}, g\pair{x}} : \oc A \multimap B \otimes B}{
			\infer[]{
				[f : A \multimap B], [g : A \multimap B], \pair{x' : \oc A}
				\vdash \caseb{x'}{\oc x \to \pair{f\pair{x}, g\pair{x}} : B \otimes B}
			}{
				\infer[]{
					\pair{x' : \oc A}, [f : A \multimap B], [g : A \multimap B]
				\vdash \caseb{x'}{\oc x \to \pair{f\pair{x}, g\pair{x}} : B \otimes B}
				}{
					\pair{x' : \oc A} \vdash x' : A &&
					\infer[]{[f : A \multimap B], [g : A \multimap B], [x : A]\vdash \pair{f\pair{x}, g\pair{x}} : B \otimes B}{
						\infer[]{\vdots}{
						\infer[]{[x : A], [f : A \multimap B], [x : A], [g : A \multimap B]\vdash \pair{f\pair{x}, g\pair{x}} : B \otimes B}{
								\infer[]{[x : A], [f : A \multimap B]\vdash f\pair{x} : B}{[f : A \multimap B] \vdash f : A \multimap B}
								&&
								\infer[]{[x : A], [g : A \multimap B]\vdash g\pair{x} : B}{[g : A \multimap B] \vdash g : A \multimap B}
							}
						}
					}
				}
			}
		}
	\]

	В функции $h$ обьект $A$ был интуционистким, но функции $f$ и $g$ работают с ним, как с линейным.
	Как мы видим линейный тип не дает нам размножать обьект (получить несколько ссылок на него в разных местах),
	но при этом у нас нет гарантий того, что до этого этот обьект не был размножен.
\end{example}

\subsection{\texorpdfstring{Уникальные типы}{Unique types}}

\newcommand{\ruleVarU}[0]{\mathtt{Var^\odot}}
\newcommand{\ruleVarS}[0]{\mathtt{Var^\otimes}}
\newcommand{\ruleAbs}[0]{\mathtt{Abs}}
\newcommand{\ruleApp}[0]{\mathtt{App}}

\begin{definition}[Род]
	Родом будем называть выражение, удовлетворяющие следующей граммтике:
	%\begin{bnf}
	\begin{alignat*}{3}
		\kappa ::= &\ \comb T && \qquad \text{Базовый тип} \\
				 | &\ \comb U && \qquad \text{Аттрибут уникальности} \\
				 | &\ * && \qquad \text{Тип} \\
				 | &\ \kappa \to \kappa
	\end{alignat*}
	%\end{bnf}
\end{definition}
\begin{definition}[Типовые константы]\ \\
	\begin{tabular}{llll}
		$\mathtt{Int}$, $\mathtt{Bool}$ & :: & $\comb T$ & Базовые типы \\
		$\to$                   & :: & $* \to * \to \comb T$ & Коструктор функций \\
		$\bullet, \times$               & :: & $\comb U$ & Уникальный, не уникальный \\
		$\vee, \wedge$                  & :: & $\comb U \to \comb U \to \comb U$ & Дизьюнкия, коньюнкция аттрибутов \\
		$\neg$                          & :: & $\comb U \to \comb U$ & Отрицание аттрибута \\
		$\mathtt{Attr}$                 & :: & $\comb T \to \comb U \to *$ & Конструктор типа
	\end{tabular}
\end{definition}
С помощью типовых констант мы можем делать различные типы разных родов.
Тип (рода $*$) состоит из базового типа (рода $\comb T$) и аттрибута уникальности (это тоже тип, только рода $\comb U$).
Аттрибут показывает обязан ли обьект этого типа быть уникальным (всего на этот обьект может существовать только одна ссылка) "--- $\bullet$,
или это не обязательно "--- $\times$.
На аттрибутах заданы различные булевы операции, которые выполняются так, будто $\bullet \equiv true$, $\times \equiv false$.

Немного сахара:
\[
	\attr{t}{u} \equiv t^u \qquad \attr{(a \to b)}{u} \equiv a \xrightarrow{u} b
\]

Грамматика выражений будет похожа на классическую грамматику выражений, за исключением того,
что у переменных будет указываться аттрибут ($x^\odot$), если на переменную есть только 1 ссылка,
или аттрибут ($x^\otimes$), если на переменную имеется несколько ссылок.
\begin{bnf}
\[
	e ::= \lambda x.e | ee | x^\odot | x^\otimes
\]
\end{bnf}%
В доказательствах будем использовать следующую нотацию: $\Gamma \vdash e : \tau |_{\fv}$.
$\Gamma$ и $\fv$ отображают переменные в типы.
Разница заключается в том, что $\Gamma$ отображает в типы рода $*$,
а $\fv$ отображает в типы рода $\comb U$.

Введём правила вывода. Эти правила похожи на правила из системы типов Хиндли-Милнера,
за исключением того, что они будут также нести информацию о уникальности типов и переменных.
\[
	\infer[\ruleVarU]{\Gamma, x : t^u \vdash x^\odot : t^u |_{x:u}}{} \qquad
	\infer[\ruleVarS]{\Gamma, x : t^\times \vdash x^\otimes : t^\times |_{x:\times}}{}
\]
В обоих правилах мы вводим в контекст новую переменную.
В $\ruleVarU$ мы вводим уникальную перемменную, а в $\ruleVarS$ - расшаренную.
Заметим, что в $\ruleVarU$ хоть переменная и уникальна, тип ее может быть и не уникальным,
так как потом эта переменная может перестать быть уникальной (например, её скопировали в процессе программы).
В $\ruleVarS$ тип переменной обязан быть не уникальным.
\[
	\infer[\ruleAbs]
		{\Gamma \vdash \lambda x.e : a \xrightarrow{\vee \fv'} b |_{\fv'}}
		{\Gamma, x : a \vdash e : b |_{\fv} && \fv' \equiv \fuck_x \fv} \\
\]

Операция $\fuck_x \fv$ выкидывает из отображения $\fv$ все пары $x : \comb U$.
Запись $\vee \fv'$ означает дизьюнкцию всех типов из отоборажения $\fv'$.

Рассмотрим правило $\ruleAbs$.
В замыкании функции $\lambda x.e$ находятся все свободные переменные из выражения $e$, за исключением связанной переменной $x$.
Если в замыкании функции находится уникальная переменная, то функция должна быть уникальна, следовательно,
атрибут уникальности функции вычисляется так: $\vee \left( \fuck_x \fv \right)$, что и написано в правиле.
\[
	\infer[\ruleApp]
		{\Gamma \vdash e_1e_2 : b |_{\fv_1 \cup \fv_2}}
		{\Gamma \vdash e_1 : a \xrightarrow{u} b |_{\fv_1} \qquad \Gamma \vdash e_2 : a |_{\fv_2}}
\]
Тут обычная аппликация Хиндли-Милнера, в которой мы запоминаем уникальность свободных переменных из обоих ветвей.
Уникальность функции нам не интересна так как мы можем вычислять функции вне зависимости от их уникальности.

\begin{example}
	Рассмотрим функцию $\mathtt{dup} \equiv \lambda x.\pair{x^\otimes, x^\otimes}$ типа $a^\times \xrightarrow{\times} a^\times \with^u a^\times$.
	Покажем, что такая функция существует в нашей системе типов:
	\[
		\infer[\ruleAbs]
			{\vdash \lambda x.\pair{x^\otimes, x^\otimes} :a^\times \xrightarrow{\times} a^\times \with^u a^\times |}
			{
				\infer[]
					{x : a^\times \vdash \pair{x^\otimes, x^\otimes} : a^\times \with^u a^\times | x : \times}
					{
						\infer[\ruleVarS]{x : a^\times \vdash x^\otimes : x : \times}{} \qquad
						\infer[\ruleVarS]{x : a^\times \vdash x^\otimes : x : \times}{} 	
					}
			}
	\]
	Так как функция $\mathtt{dup}$ дублирует аргументы, она принимает переменную неуникального типа.
	Компилятор это может легко проверить, так как в теле функции есть 2 ссылки на $x$,
	и именно поэтому она помечена как $\otimes$.
	Также функция $\mathtt{dup}$ полиморфна по аттрибуту уникальности возвращаемого значения,
	так как мы можем попросить у функции как и уникальную пару, так и расшаренную.
\end{example}
\begin{example}
	Давайте по другому определим функцию
	$\mathtt{dup'}~x \equiv \left(\lambda f. \pair{f^\otimes \perp, f^\otimes \perp} \right) \left(\mathtt{const}~x^\odot \right)$.
	Очевидно, она делает тоже самое, что и функция $\mathtt{dup}$.
	
	Давайте установим тип функции $\mathtt{const}$.
	Сначала запишем ее тип без указания аттрибутов уникальности над стрелочками: $t^u \to s^v \to t^u$.
	Если мы частично применим функцию $\mathtt{const}$, передав в нее переменную уникального типа,
	то мы должны получить функцию $(\mathtt{const}~x)$ уникального типа.
	Если бы это было не так, то мы смогли бы полученную функцию применить в нескольких местах, и,
	следственно, смогли получить несколько ссылок на результат функции.
	Но это противоречит типовой системе, так как результат (он же первый аргумент исходной функции) "--- уникален.

	Значит, полный тип функции $\mathtt{const}$ равен $t^u \xrightarrow{\times} s^v \xrightarrow{u} t^u$.

	В первой части функции $\mathtt{dup'}$ лямбда-выражение принимает расшаренную функцию.
	То есть функция $(\mathtt{const}~x)$ - не уникальна. Значит тип переменной $x$ не уникален.
	Это является примером того, когда уникальная переменная имеет неуникальный тип.
\end{example}
